%% start of file `template.tex'.
%% Copyright 2006-2011 Xavier Danaux (xdanaux@gmail.com).
%
% This work may be distributed and/or modified under the
% conditions of the LaTeX Project Public License version 1.3c,
% available at http://www.latex-project.org/lppl/.


\documentclass[11pt,a4paper]{moderncv}

% moderncv themes
\moderncvtheme[blue]{classic}                  % optional argument are 'blue' (default), 'orange', 'green', 'red', 'purple', 'grey' and 'roman' (for roman fonts, instead of sans serif fonts)
%\moderncvtheme[green]{classic}               % idem

% character encoding
\usepackage[utf8]{inputenc}                   % replace by the encoding you are using
%\usepackage{CJKutf8}                         % if you need to use CJK to typeset your resume in Chinese, Japanese or Korean

% adjust the page margins
\usepackage[scale=0.8]{geometry}
%\setlength{\hintscolumnwidth}{3cm}           % if you want to change the width of the column with the dates
%\setlength{\maketitlenamewidth}{10cm}}       % only for the classic theme, if you want to change the width of your name placeholder (to leave more space for your address details


% personal data
\firstname{Ravikiran}
\familyname{Janardhana}
\address{1100 W NC HWY 54 BYP, APT 38B}{CHAPEL HILL NC-27516}    % optional, remove the line if not wanted
\mobile{919-448-8740}                    % optional, remove the line if not wanted
\email{ravikirn@cs.unc.edu}                      % optional, remove the line if not wanted
%%\homepage{www.ravikiranj.net}                % optional, remove the line if not wanted
%\photo[64pt]{ravikirn.jpg}                  % '64pt' is the height the picture and 'picture' is the name of the picture file; 

%\nopagenumbers{}                             % uncomment to suppress automatic page numbering for CVs longer than one page
%----------------------------------------------------------------------------------
%            content
%----------------------------------------------------------------------------------
\begin{document}
%\begin{CJK*}{UTF8}{gbsn}                    % to typeset your resume in Chinese using CJK
% format name size to adjust for long length and color
\renewcommand*{\firstnamefont}{\fontsize{30}{26}\sffamily\mdseries\upshape}
\renewcommand*{\familynamefont}{\firstnamefont}
\definecolor{familynamecolor}{rgb}{0.65,0.65,0.65}
\definecolor{firstnamecolor}{rgb}{0.45,0.45,0.45}

\maketitle
\section{Experience}
\cventry{2011--present}{Research Assistant}{University of North Carolina at Chapel Hill}{Chapel Hill}{}{
\begin{itemize}
\item{Worked on identifying fiber crossings in white matter of the brain under \textit{Dr. Martin Styner}}
\item{The main goal of my research is to find the fiber crossing landmarks and use it for registration of human brains}
\end{itemize}}

\vspace{1mm}

\cventry{2009--2011}{Software Engineer}{Yahoo! India}{Bangalore}{}{
\begin{itemize}
\item{Developed Facebook, Twitter and LinkedIn modules} 
\item{Implemented front-end based instrumentation of My Yahoo! product in order to compute page-views, module-views and click through rates (\textit{CTR})}
\item{Developed an instrumentation dashboard to showcase daily statistics of views and clicks utilizing Apache Hadoop and Pig for the backend to query the records (\textit{in the order of millions})}
\item{Demonstrated the quality of a utility player by handling both the frontend as well as the backend tasks which included porting legacy code, improving existing modules and developing new modules}
\end{itemize}}

\vspace{2mm}

\cventry{2007}{Teaching Assistant}{PES Institute of Technology}{Bangalore}{}{
\begin{itemize}
\item{Worked as teaching assistant in the course of \textit{Advanced Microprocessors}}
\item{Developed assembly programs in Microsoft Assembler (MASM) to demonstrate binary search, bubble sort, string operations and simple calculator and taught the same}
\end{itemize}}

\section{Education}
\cventry{2011--present}{MS.}{University of North Carolina at Chapel Hill}{Chapel Hill, NC}{}{Masters in Computer Science}
\cventry{2005--2009}{BE.}{Peoples Education Society Institute of Technology (PESIT)}{Bangalore, India}{Bachelor of Engineering in Computer Science}{Percentage: 88.40, \textbf{University Topper}}

\section{Computer skills}
\cvline{Programming}{C/C++, PHP, Python, Java, Perl}
\cvline{Web}{HTML5, Javascript, CSS, YUI2/3, jQuery}
\cvline{Data Mining}{Pig, Hadoop}
\cvline{Database}{MySQL, Oracle}
\cvline{Platforms}{Linux, Web}
\cvline{Concepts}{Motion Planning for Robots, Computer Vision, Digital Image Processing, Operating Systems, Algorithms and Data Structures}

\vspace{9mm}

\section{Academic Projects}
\cventry{2011}{Roadmap-based Motion Planning in Dynamic Environments}{}{}{}{
\begin{itemize}
\item {Implemented a motion planning algorithm for a point robot to navigate in a dynamic environment consisting of both static and dynamic moving obstacles from start to goal}
\end{itemize}
}

\vspace{2mm}

\cventry{2011}{Identifying fiber crossing landmarks in the white matter of the brain}{}{}{}{
\begin{itemize}
\item {Designed and Implemented an algorithm to identify fiber crossing landmarks in the white matter using entropy, fiber segments per voxel and fiber orientation dispersion}
\item {The input for the algorithm is a Diffusion Weighted MR Image (DWI) and the output is an image which highlights the fiber crossing landmarks}
\end{itemize}
}

\vspace{2mm}

\cventry{2009}{Track Me - A suite of innovative user interfaces}{}{}{}{
\begin{itemize}
\item {Track Me is a series of innovative user interfaces whose goal is to help users interact with their PC in a natural manner. It consists of:
    \begin{itemize}
       {\setlength\itemindent{25pt}  \item{Fintrack ME - Finger Tracking Mouse Emulator}}
       {\setlength\itemindent{25pt}  \item{Talk2me - A speech driven Powerpoint and Windows Media Player assistant}}
       {\setlength\itemindent{25pt}  \item{Point2me - A laser point tracking Powerpoint and Windows Media Player assistant}}
    \end{itemize}
\item {This won the best project award in the Department of Computer Science at Prakalpa 2009 organized by PES Institute of Technology}
}
\end{itemize}
}

\vspace{2mm}

\cventry{2008 -- 2009}{American Sign Language Interpreter}{}{}{}{
\begin{itemize}
\item {Developed a real-time interpreter of American Sign Language alphabets which converts hand gestures into text, which is further read out by a speech engine}
\item {This project resulted in a research paper [1] which was presented at \textit{International MultiConference of Engineers and Computer Scientists 2009, Hong Kong}}
\item {An extension of this work appeared as a book chapter [2] in \textit{Intelligent Automation and Computer Engineering}, Springer, 321-332, 2010}
\end{itemize}
}

\section{Awards and Achievements}
\cventry{Mar 2011}{Promoted}{Yahoo! India}{Bangalore, India}{}{Promoted to Senior Software Engineer}
\cventry{Jan 2010}{University Gold Medal}{Visvesvaraya Technological University}{PESIT}{Belgaum, India}{Awarded Gold Medal for being the University topper in Bachelor of Engineering (B.E) in Computer Science (2005-09) }
\cventry{Jul 2009}{Certificate of Merit}{International MultiConference of Engineers and Computer Scientists 2009}{Hong Kong}{}{Awarded Certificate of Merit for the conference paper \textit{Finger Detection for Sign Language Recognition} presented at IMECS 2009, Hong Kong}

\section{Interests}
\cvline{Software}{Linux and Open Source}
\cvline{Web/Mobile}{HTML5, Android and iOS App Development}
\cvline{Sports}{Football, Cricket, Basketball}
\cvline{Entertainment}{Classical/Electric Guitar, Keyboard, Computer Games and Table Tennis}

% Publications from a BibTeX file without multibib\renewcommand*{\bibliographyitemlabel}{\@biblabel{\arabic{enumiv}}}% for BibTeX numerical labels
\nocite{*}
\bibliographystyle{plain}
\bibliography{publications}                  % 'publications' is the name of a BibTeX file

%\clearpage\end{CJK*}                        % if you are typesetting your resume in Chinese using CJK; the \clearpage is required for fancyhdr to work correctly with CJK, though it kills the page numbering by making \lastpage undefined
\end{document}

%% end of file `template_en.tex'.
